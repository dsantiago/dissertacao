\begin{resumo}
% Este trabalho consiste em unificar abordagens contemporâneas na avaliação da cardiomiopatia. Como apoio da análise radiômica, a qual extrai-se informações das características estatísticas e de textura de uma imagem médica e de características oriundas de uma rede neural clássica para visão computacional, como a \textit{ResNet50}, é possível obter resultados promissores. Os resultados certificam que a união de informações de diversos âmbitos, a cerca de um dado paciente, quando aliadas, podem culminar em resultados mais interessante se comparado aos dados de forma isolada. O presente trabalho visa, usar as abordagens citadas, baseado em literaturas prévias, efetuando uma aplicação inédita para o teste de cardiomiopatia, adaptando e propondo uma arquitetura mais robusta de forma obter bons resultados.

A crescente disponibilidade de exames de imagem médica, como a ressonância magnética, gera um grande volume de dados, tornando sua análise complexa e desafiadora. Neste cenário, abordagens computacionais avançadas podem otimizar a interpretação dessas imagens e auxiliar no diagnóstico precoce de doenças cardiovasculares. Este trabalho consiste em unificar abordagens contemporâneas na avaliação da cardiomiopatia. Com o apoio da análise radiômica, a qual extrai informações das características estatísticas e de textura de uma imagem médica, e de características oriundas de uma rede neural clássica para visão computacional, como a \textit{ResNet50}, é possível obter resultados promissores. Os resultados certificam que a união de informações de diversos âmbitos, acerca de um dado paciente, quando aliadas, podem culminar em resultados mais interessantes se comparados aos dados de forma isolada. O presente trabalho visa usar as abordagens citadas, baseado em literaturas prévias, efetuando uma aplicação inédita para o teste de cardiomiopatia, adaptando e propondo uma arquitetura mais robusta de forma a obter bons resultados.



\palavraschave{Radiomics, Mecanismo de Atenção, Transformers, Cardiomiopatia}
 
\end{resumo}