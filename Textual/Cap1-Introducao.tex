%% USPSC-Introducao.tex

% ----------------------------------------------------------
% Introdução (exemplo de capítulo sem numeração, mas presente no Sumário)
% ----------------------------------------------------------
\chapter[Introdução]{Introdução}
\label{chap:introducao}
\vspace{-\baselineskip} %Manter para garantir o espaçamento da biblioteca.

A tecnologia está cada vez mais presente nas diversas áreas do conhecimento, trazendo uma infinidade de benefícios e facilitando o cotidiano contemporâneo. Entre os diversos campos impactados, a área médica destaca-se como uma das que mais se beneficiaram da inovação tecnológica. Desde meados dos anos 2000, a quantidade de dados gerados na medicina tem crescido exponencialmente, atingindo projeções de milhares de \textit{exabytes} a partir de 2020 \cite{gantzDIGITALUNIVERSE2020}. Esse cenário evidencia a importância de ferramentas que possam processar e analisar eficientemente grandes volumes de informações.

Exames de imagem, como a \gls{tc} e a \gls{rmc}, tornaram-se essenciais na medicina moderna. Esses exames não apenas oferecem uma representação tridimensional detalhada de estruturas do corpo humano, mas também produzem dados que podem ser analisados de forma quantitativa. Em paralelo, a \gls{ia} trouxe avanços significativos à análise de imagens diagnósticas, proporcionando maior eficiência e precisão nos diagnósticos médicos \cite{argentieroApplicationsArtificialIntelligence2022}.

Nesse sentido, as redes neurais profundas, um dos principais pilares da \gls{ia}, têm demonstrado alto desempenho em tarefas de visão computacional, como classificação de imagens, detecção de objetos e segmentação. Essas redes conseguem aprender características discriminantes uma vez otimizadas a cerca do conjunto de dados em que foi treinada. Além disso, arquiteturas como o \textit{transformers} ficaram populares por serem comumente usadas em redes generativas auto-regressivas para geração sintética de texto, também conhecidas como \gls{llm}, tendo como seu exemplo mais conhecido o \textit{ChatGPT}. Os \textit{transformers} são arquiteturas que destacam-se pela capacidade de paralelismo e pelo uso de mecanismos de autoatenção, que permitem ao modelo focar nas partes mais relevantes dos dados de entrada \cite{russell2020artificial}.

Adicionalmente, técnicas de processamento de imagem como a análise de textura já vem sendo utilizada por várias décadas em diversos domínios da medicina. A análise radiômica emergiu como uma ferramenta poderosa na extração de informações quantitativas de imagens médicas, capturando padrões que muitas vezes passam despercebidos ao olho humano. Essa abordagem tem mostrado potencial em diversas áreas, como oncologia e cardiologia \cite{schofieldTextureAnalysisCardiovascular2019a}.

No domínio da análise de imagens cardíacas, a análise de textura aplicada à \gls{rmc} tem avaliado o risco de arritmia pós-infarto do miocárdio. O uso de análise de textura para \gls{rmc} em imagens sem contraste e com realce tardio de gadolínio em pacientes com cardiomiopatia para prever o resultado do exame é uma área particular de interesse \cite{schofieldTextureAnalysisCardiovascular2019a}.

A \gls{cmh} é uma das cardiomiopatias mais comuns, frequentemente diagnosticada em jovens e indivíduos de meia-idade. Embora em muitos casos seja assintomática, a doença pode levar a condições graves, como insuficiência cardíaca e acidente vascular cerebral. Isso torna o diagnóstico precoce essencial para prevenir desfechos adversos \cite{kwonComparisonMortalityCause2022}. Neste sentido a análise radiômica que consiste em extrair dados qualitativos de imagens médicas, incluindo, em muitos casos, a análise da textura dessas imagens \cite{lambinRadiomicsExtractingMore2012}. A análise radiômica pode auxiliar com o diagnóstico prévio afim de compreender e atuar nos casos de cardiomiopatia que demonstrem risco ao paciente.

Neste cenário, combinar técnicas de \gls{ia} e análise radiômica representa uma estratégia promissora para a detecção de cardiomiopatias entre outras condições cardíacas. Estudos recentes, como o de \citeonline{aiSelfAttentionBasedFusion2023}, demonstraram que a integração de características profundas e radiômicas podem melhorar significativamente o desempenho preditivo de modelos diagnósticos de câncer de pulmão via imagens de \gls{tc}. Esses modelos baseiam-se em mecanismos de autoatenção para identificar padrões relevantes em dados concatenados, alcançando acurácia de até $82,35\%$ e \gls{auc} de $0,74$.

Assim, o presente trabalho propõe a implementação e validação de uma estratégia de fusão que combine características radiômicas e profundas, com o uso de mecanismos de autoatenção, para melhorar a classificação de cardiomiopatias. Além de avançar o estado da arte em diagnósticos médicos, esta pesquisa busca contribuir para a adoção de soluções mais eficazes e acessíveis no apoio à decisão clínica.

%---------------------------------------------------------
\section{OBJETIVOS}
\label{sec:cap1_objetivo}

Os objetivos deste trabalho consistem em propor e validar uma abordagem inovadora para a classificação de cardiomiopatias utilizando técnicas de inteligência artificial e análise radiômica. Estes objetivos foram definidos para responder às necessidades clínicas e avançar o estado da arte na área de diagnóstico médico. Com isso, o objetivo geral do presente projeto é desenvolver, implementar e validar um modelo de classificação de cardiomiopatias, utilizando a integração de características radiômicas e profundas mediadas por mecanismos de autoatenção. Os objetivos específicos, por sua vez, são elencados conforme os itens abaixo:


\begin{enumerate}
\item Identificar e extrair características radiômicas de imagens de \gls{rmc}, capturando padrões texturais e estatísticos relevantes para a classificação das cardiomiopatias.

\item Projetar um pipeline de análise que combine eficientemente características radiômicas e profundas, garantindo a fusão informativa para modelos de aprendizado profundo.

\item Implementar um modelo baseado em redes neurais profundas que utilize mecanismos de atenção para priorizar regiões relevantes nas imagens, melhorando a acurácia e a interpretabilidade do modelo.

\item Validar a eficácia do modelo proposto utilizando métricas padrão, como acurácia, precisão, revocação, F1-score e \gls{auc}, em um conjunto de dados públicos e diversificado.

\item Comparar o desempenho do modelo proposto com técnicas existentes na literatura, identificando avanços e limitações em relação às abordagens tradicionais.
\end{enumerate}

%---------------------------------------------------------
\section{ESTRUTURA DO TRABALHO}
\label{sec:cap1_estrutura_trabalho}

Este trabalho está organizado em sete capítulos, cada um projetado para abordar diferentes aspectos da pesquisa e sua implementação. A seguir, apresenta-se a estrutura detalhada desta pesquisa:

O \textbf{Capítulo \ref{chap:introducao}} apresenta o contexto geral do trabalho, destacando a relevância do uso de inteligência artificial e análise radiômica no diagnóstico de cardiomiopatias. São expostos os objetivos gerais e específicos da pesquisa, bem como a motivação para o desenvolvimento deste estudo. O \textbf{Capítulo \ref{chap:fundamentacao_teorica}} discute os conceitos teóricos que embasam o trabalho, incluindo princípios de redes neurais profundas, análise radiômica, mecanismos de autoatenção e sua aplicação em imagens médicas. Este capítulo também revisa estudos relacionados que contribuíram para o estado da arte na área. No \textbf{Capítulo \ref{chap:trab_relacionados}} é estudado  os trabalhos recentes, não superior a cinco anos passados, de outros autores que estão de alguma forma relacionados com tema, sendo estes trabalhos relacionados oriundos de uma minuciosa revisão sistemática da literatura. O \textbf{Capítulo \ref{chap:metodologia}} detalha o método proposto para a classificação de cardiomiopatias, incluindo a descrição do conjunto de dados utilizado, o processo de extração de características radiômicas, o desenvolvimento do modelo baseado em redes neurais profundas e as etapas de validação e experimentação. O \textbf{Capítulo \ref{chap:proposta_experimental}} faz a proposição experimental que contempla os dados utilizados, informações do seu pré-processamento, hiperparâmetros planejados e demais informações a cerca do experimentos realizados. Neste capítulo também é apresentado os resultados obtidos com a implementação do modelo proposto, acompanhados de análises quantitativas e qualitativas. Discute-se o desempenho do modelo em comparação com outras abordagens e suas implicações técnicas. Finalmente, o \textbf{Capítulo \ref{chap:cap7_conclusao}} sintetiza os principais achados da pesquisa, destacando as contribuições do trabalho para o estado da arte e suas potenciais aplicações. Além disso, são sugeridas direções para estudos futuros que possam expandir e aprimorar as técnicas apresentadas.
